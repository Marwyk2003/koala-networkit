\section{Overview}

This section provides a brief historical overview of the development of priority queue data structures. We highlight several widely used and theoretically significant variants, setting the stage for the implementations and comparisons that follow. At the end of the section, we also provide a table summarizing the time complexities of priority queue operations on various implementations.

\subsection{Classical approaches}

The most straightforward, albeit inefficient, implementations of priority queues use either unsorted or sorted arrays (or lists). An unsorted array allows constant-time insertions, but locating and removing the element with the lowest priority requires scanning the entire collection, resulting in \(O(n)\) time for both \texttt{pop} and \texttt{peek}. In contrast, a sorted array offers constant-time access to the minimum element, but insertion becomes costly, taking \(O(n)\) time to maintain the sorted order.

The binary heap, introduced by J.W.J. Williams in 1964 \cite{williams1964heapsort}, is the most commonly used priority queue in both academic and practical settings. A binary heap (specifically, a \emph{min}-heap in our context) is a complete binary tree that satisfies the heap-order property: each node is less than or equal to its children. Represented implicitly in an array, it supports both insertion and minimum deletion in \(O(\log n)\) time and offers constant-time access to the minimum via \texttt{peek}.

Introduced by Fredman and Tarjan in 1987 \cite{fredman1987fibonacci}, the Fibonacci heap consists of a forest of heap-ordered trees and is one of the earliest and best-known examples of amortized analysis. Most operations run in \(O(1)\) amortized time, while \texttt{pop} takes \(O(\log n)\) amortized time. Although asymptotically optimal in the amortized sense (as improving these bounds would violate known lower bounds for sorting), Fibonacci heaps are rarely used in practice due to their complex implementation and poor constant factors.

In contrast to amortized-efficient structures, worst-case optimal priority queues guarantee tight bounds on every operation. The two most notable examples are the \emph{Brodal queue} and the \emph{Strict Fibonacci heap}. These data structures support all operations except \texttt{pop} in \(O(1)\) worst-case time; \texttt{pop} itself requires \(O(\log n)\) worst-case time.

The Brodal queue, introduced by Gerth Stølting Brodal in 1996 \cite{brodal1996worst}, was the first structure to achieve these bounds. The Strict Fibonacci Heap, proposed by Kaplan, Tarjan, and Zwick in 2002 \cite{kaplan2002fibonacci}, builds on ideas from the original Fibonacci heap while enforcing worst-case performance. Despite their optimal theoretical guarantees, both structures are known for their extreme implementation complexity and poor practical performance. As such, they are seldom used in practice and are not included in this work.

The following table summarizes the asymptotic time complexity of priority queue implementations mentioned above. A dash (\(-\)) indicates that a structure does not support the corresponding operation efficiently. An asterisk (\(^*\)) denotes amortized complexity.

\begin{table}[h!]
\centering
\begin{tabular}{@{}lccccc@{}}
\toprule
\textbf{Structure}       & \texttt{push}          & \texttt{pop}           & \texttt{peek}          & \texttt{decreaseKey}     & \texttt{meld} \\
\midrule
Unsorted array/list      & \(O(1)\)               & \(O(n)\)               & \(O(n)\)               & -                         & -             \\
Sorted array/list        & \(O(n)\)               & \(O(1)\)               & \(O(1)\)               & -                         & -             \\
Binary heap              & \(O(\log n)\)          & \(O(\log n)\)          & \(O(1)\)               & \(O(\log n)\)             & -             \\
Fibonacci heap           & \(O(1)^*\)             & \(O(\log n)^*\)        & \(O(1)\)               & \(O(1)^*\)                & \(O(1)^*\)     \\
Brodal/Strict Fibonacci & \(O(1)\)              & \(O(\log n)\)          & \(O(1)\)               & \(O(1)\)                  & \(O(1)\)       \\
\bottomrule
\end{tabular}
\caption{Time complexities of classical priority queue implementations.}
\end{table}

\subsection{Our implementations}

We begin by describing several priority queue data structures that operate under a specific assumption: the key set consists of integers drawn from a bounded universe. We refer to these as \emph{integer priority queues}. Their performance, both in time and space, depends on the size \(M\) of this universe. The table below summarizes the complexity of various integer priority queue implementations. Since none of the listed structures efficiently support \texttt{decreaseKey} or \texttt{meld}, we omit those operations from the table.

\begin{table}[h!]
\centering
\begin{tabular}{@{}lcccc@{}}
\toprule
\textbf{Structure}       & \texttt{push}          & \texttt{pop}           & \texttt{peek}          & \texttt{space} \\
\midrule
van Emde Boas tree       & \(O(\log \log M)\)     & \(O(\log \log M)\)     & \(O(1)\)               & \(O(M)\) \\
X-fast trie              & \(O(\log M)\)          & \(O(\log M)\)          & \(O(1)\)               & \(O(n \log M)\) \\
Y-fast trie              & \(O(\log \log M)\)     & \(O(\log \log M)\)     & \(O(1)\)               & \(O(n)\) \\
\bottomrule
\end{tabular}
\caption{Time and space complexities of integer priority queue implementations.}
\end{table}

In a later chapter, we describe several \emph{general-purpose priority queues}. These structures support arbitrary key sets and can be viewed as alternatives to classical implementations. Their complexities are summarized in Table~1.3.

\vspace{1em}

\begin{table}[h!]
\centering
\begin{tabular}{@{}lccccc@{}}
\toprule
\textbf{Structure}       & \texttt{push}          & \texttt{pop}           & \texttt{peek}          & \texttt{decreaseKey}     & \texttt{meld} \\
\midrule
Weak heap                & \(O(\log n)\)          & \(O(\log n)\)          & \(O(1)\)               & -                         & -               \\
Skew heap                & \(O(\log n)^*\)        & \(O(\log n)^*\)        & \(O(1)\)               & \(O(\log n)^*\)            & \(O(\log n)^*\) \\
Rank-pairing heap        & \(O(1)\)               & \(O(\log n)^*\)        & \(O(1)\)               & \(O(1)^*\)                 & \(O(1)\)        \\
\bottomrule
\end{tabular}
\caption{Time complexities of implemented general-purpose priority queues.}
\end{table}
